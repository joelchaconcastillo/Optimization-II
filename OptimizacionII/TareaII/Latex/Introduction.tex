\IEEEPARstart{P}{rogramación} lineal tiene un número de funciones objetivo y restricciones lineales, las cuales pueden estar conformadas tanto por igualdades como desigualdades.
%
El conjunto factible es un polítopo, el cual es un conjunto convexo conectado con caras poligonales planas \cite{nocedal2006sequential}. 
%
Los programas lineales usualmente son establecidos y analizados en la siguiente forma estándar:



La forma estándar de un problema de programación lineal es:
\begin{equation}\label{estandar}
min \quad \vec{c}^T \vec{x}, \quad sujeto \quad a: \quad A\vec{x} = \vec{b}, \quad \vec{x} \geq 0
\end{equation}

La forma dual equivalente es representado de la forma:
\begin{equation}\label{dual}
max \quad \vec{b}^T \vec{\pi}, \quad sujeto \quad a: \quad A^T \vec{\pi} \leq \vec{c}
\end{equation}

\subsection{Condiciones de optimalidad}
\begin{equation}
\begin{split}
\mathcal{L}(x, \pi, s) =& c^T x - \pi^T ( Ax - b) - s^T x \\
sujeto \quad a \\
A^T \pi + s =& c \\
Ax =& b \\
x \geq& 0 \\
s \geq& 0 \\
s_i x_i =& 0, \quad i=1,2,...,n
\end{split}
\end{equation}
donde $\pi$ es el coeficiente de Lagrange.
